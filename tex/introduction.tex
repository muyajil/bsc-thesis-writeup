\section{Introduction}
To protect the confidentiality of sensitive information stored in databases, it is essential to restrict access to the data.
%
For this, the SQL standard defines an access control model, and existing Database Management Systems (DBMSs) have accordingly implemented it. 
%
One of the main tasks of database access control mechanisms is to determine whether \texttt{SELECT} queries issued by the users are secure.
%
This is \emph{essential} to effectively secure database systems and prevent the leakage of sensitive information.

The access control mechanism proposed in the SQL standard, and implemented in the PostgreSQL DBMS, is rather simple.
%
It just relies on syntactic checks to determine whether the submitted \texttt{SELECT} query is secure.
%
In more detail, a \texttt{SELECT} query $q$ is deemed secure if the user issuing $q$ is authorized to read, according to the access control policy, all the tables and views referenced in $q$. % are readable to , according to the access control policy.
%
This mechanism is overly restrictive.
%
Indeed, as already identified in~\cite{rizvi2004extending, guarnieri2014optimal, wang2007correctness,zhang2005authorizations}, even queries not satisfying this criterion are secure.

Recently, Guarnieri et al.~\cite{guarnieri2016strong} investigated the limitations of SQL's access control model and accordingly  developed a novel, more secure, access control mechanism.
%
The main building block of this mechanism is an algorithm for checking the security of \texttt{SELECT} queries.
%
This algorithm is used not only to decide whether \texttt{SELECT} queries are authorized or not, but also to determine whether all the other commands may leak sensitive information. 
%
Using naive algorithms, such as those proposed in the SQL standard, for this would result in an extremely unusable system, due to their restrictiveness.
%
Motivated by this, Guarnieri et al. developed a more permissive algorithm, based on query rewriting, for checking the security of \texttt{SELECT} queries.

The algorithm proposed in~\cite{guarnieri2016strong} however, has two main practical limitations.
%
First, it is based on the relational calculus, whereas existing DBMSs are based on SQL. Despite SQL being inspired by the relational calculus it differs from it in significant ways. For instance, it has a different syntax and semantics, e.g., SQL uses a bag semantics whereas the relational calculus relies on a set-based semantics.
%
Second, it is defined primarily for boolean queries, whereas \texttt{SELECT} queries usually retrieve tuples.
%
Even though boolean queries can be used to encode the result of non-boolean queries, this encoding may introduce a exponential blow-up in the formula's size.

Motivated by the idea of extending the mechanism from~\cite{guarnieri2016strong} to the entire SQL language, in this thesis we develop a novel algorithm for checking the security of \texttt{SELECT} queries.
%
Our algorithm extends the one proposed by Guarnieri et al. to the SQL language and is more permissive than the one proposed in the SQL standard and  implemented in PostgreSQL.

\smallskip
\noindent
{\bf Contributions.}
%
Guarnieri et al. have tested the security of access control mechanisms in various DBMSs in~\cite{guarnieri2016strong}.
%
According to their findings PostgreSQL is the most vulnerable DBMS, therefore we have decided that implementing the new access control mechanism for \texttt{SELECT} queries.

As we will see in Chapter 4 the current architecture of PostgreSQL makes it difficult to implement such a new algorithm.
%
Therefore first, we refactor the original PostgreSQL's implementation to isolate the access control mechanism and provide a general interface for access requests, independent of the type of query.
%
The original implementation requires a deep understanding of the backend architecture of PostgreSQL to be able to extend and improve the current access control algorithm.
%
Our refactoring provides the possibility of improving and extending access control without interfering with the internals of the DBMS.
%

Second, we extend the access control mechanism proposed by Guarnieri et al.~\cite{guarnieri2016strong} to SQL.
%
This is essential to effectively secure PostgreSQL and prevent leakages of sensitive information.
%
In more detail, we implemented a new access control mechanism for \texttt{SELECT} queries.
%
This is the first building block to successfully implement strong access control mechanisms, such as the one proposed by Guarnieri et al.

\smallskip
\noindent
{\bf Organization.}
%
In chapter 2 we will introduce some motivating examples.
%
In chapter 3 we will give background information to access control by comparing current access control mechanism to the model introduced in~\cite{guarnieri2016strong}.
%
After that we will introduce the architecture of PostgreSQL and the new access control algorithm in chapter 4. 
%
In chapter 5 we will begin by showing the formal models of all aspects of the new security defintions and how they are extended from the relational algebra to SQL.  
%
In chapter 6 we will introduce some related work to the thesis. 
%
Finally in chapter 7 the conclusions are drawn and future possibilities are discussed.