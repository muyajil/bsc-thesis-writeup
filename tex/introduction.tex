\section{Introduction}
When storing sensitive information in databases it is \emph{essential} to control access to it. 
%
For this the SQL standard defines an access control model, which has been implemented by commercial Database Management Systems (DBMSs). 
%
This access control model effectively regulates \emph{direct access}, i.e., through \texttt{SELECT} queries, to the data, it falls far short of preventing \emph{indirect access} to sensitive data.
%
Namely, attackers can exploit advanced database features, such as triggers and views, as well as the DBMS' semantics to infer sensitive information from sequences of authorized commands~\cite{guarnieri2016strong, ..., ..., ...}.
%
These attacks are not possible due to misconfigured systems or software vulnerabilities. Rather, it is due to the unrealistic security assumptions on which the SQL access control model relies, namely attackers can only infer information from \texttt{SELECT} queries' results withouth being able to infer information from other aspects of the systems behaviour.

Recently, Guarnieri et al.~\cite{guarnieri2016strong} investigated the limitations of the SQL access control model. They identified two families of attacks that attackers can carry out by exploiting advanced database features and the DBMS' semantics.
%
\emph{Integrity attacks} allow an attacker to perform non-authorized changes to the database, whereas \emph{confidentiality attacks} allow an attacker to learn sensitive information.
%
They also developed a formal framework, consisting of formal database semantics and attacker model as well as precise security properties, to reason about the precise security guarantees offered by access control mechanisms in database systems.
%
Using this framework, they developed an open access control mechanism that provably prevents both integrity and confidentiality attacks.

\remark{We need a strong motivation. Why do we need the work in this thesis? How does it improve the state-of-the-art? Why should a database administrator read the thesis and use what we propose?}

\remark{The third paragraph should clearly identify (1) the weaknesses and limitations of existing solutions (such as~\cite{guarnieri2016strong}), and (2) why they cannot be used in practice. This is \underline{the} paragraph where we say that everything is bad. Later we will show how this thesis saves the world.}

However solutions such as the one proposed by Guarnieri et al. have several limitations in practice.
%
For one the depicted model supports the relational algebra, which uses a set semantics, whereas SQL uses a bag sematics.
%
Furthermore the algorithm produced by Guarnieri et al. implemented as proposed, would produce a very large overhead, up to 200\%.
%

\remark{What are other limitations here? Until now I have only met these :)}

\remark{Should I include other proposed solutions here? If yes which ones or where do I find those?}

\smallskip
\noindent
{\bf Contributions.}
This thesis explores the possibility of implementing a solution such as the one proposed by Guarnieri et al.~\cite{guarnieri2016strong} in a state-of-the-art DBMS.
%
After mounting the attacks agains several popular DBMS it was found that PostgreSQL is the most vulnerable, failing to prevent both attack families.
%
In the course of pursuing this implementation several contributions were made.
%
First the codebase of PostgreSQL was refactored in such a way, that all access control relevant code is located in one module, making it easier for future extensions or improvements, while also allowing to implement any other access control algorithm as a plug-in for the main software.
%
Second the semantics of the algorithm from~\cite{guarnieri2016strong} was extended from the relational algebra to SQL, and therefore enabling the implementation of this algorithm into existing DBMSs. Thus it is possible to have a provably secure access control algorithm in widely used database systems.

%
\remark{Here, you should list all the contributions of the thesis.
%
For each contribution, you should explain (1) why is it important and meaningful, and (2) how it improves the state-of-the-art.
%
 Note that contributions $\neq$ list of things that you did. A contribution is something new that improves the state-of-the-art. After this section, the reader must be convinced that the work in this thesis is an important contribution to humanity :-)}


\smallskip
\noindent
{\bf Organization.}
\remark{just to quickly enumerate all chapters}
\begin{enumerate}
	\item Motivating examples
	\item Background
	\item Architecture
	\item Algorithms
	\item Experiments
	\item Related Work
	\item Conclusion/Future work
\end{enumerate}
In the next chapter we will introduce some motivating examples of attacks from both families. In chapter 3 we will give background information about the advanced features that can be exploited. After that we will introduce the architecture of PostgreSQL and the new access control algorithm in chapter 4. In chapter 5 we will begin by showing the formal models of all aspects of the new security defintions and how they are extended from the relational algebra to SQL. Furthermore we will conduct experiments with state-of-the-art benchmark databases and queries in chapter 6. In chapter 7 we will introduce some related work to the thesis. Finally in chapter 8 the conclusions are drawn and future possibilities are discussed.
\remark{In Chapter X we do A. In Chapter Y we do B. \ldots}