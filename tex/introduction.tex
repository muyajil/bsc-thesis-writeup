\section{Introduction}
When storing sensitive information in databases it is \emph{essential} to restrict access to it. 
%
For this the SQL standard defines an access control model, which has been implemented by commercial Database Management Systems (DBMSs). 
%
This access control model successfully regulates \emph{direct access}, i.e., through \texttt{SELECT} queries, to the data. It however falls far short of preventing \emph{indirect access} to sensitive data.
%
Namely, attackers can exploit advanced database features, such as triggers and views, as well as the DBMS' semantics to infer sensitive information from sequences of authorized commands~\cite{guarnieri2016strong, ..., ..., ...}.
%
The mentioned attacks are possible due to the unrealistic security assumtions on which the SQL access control model relies, namely attackers can only infer information from \texttt{SELECT} queries' results without being able to infer information from other aspects of the system's behaviour.

Recently, Guarnieri et al.~\cite{guarnieri2016strong} investigated the limitations of the SQL access control model. They identified two families of attacks that attackers can carry out by exploiting advanced database features and the DBMS' semantics.
%
\emph{Integrity attacks} allow an attacker to perform non-authorized changes to the database, whereas \emph{confidentiality attacks} allow an attacker to learn sensitive information.
%
They also developed a formal framework, consisting of formal database semantics and attacker model as well as precise security properties, to reason about the precise security guarantees offered by access control mechanisms in database systems.
%
Using this framework, they developed an open access control mechanism that provably prevents both integrity and confidentiality attacks.

\remark{We need a strong motivation. Why do we need the work in this thesis? How does it improve the state-of-the-art? Why should a database administrator read the thesis and use what we propose?}

\remark{The third paragraph should clearly identify (1) the weaknesses and limitations of existing solutions (such as~\cite{guarnieri2016strong}), and (2) why they cannot be used in practice. This is \underline{the} paragraph where we say that everything is bad. Later we will show how this thesis saves the world.}

The framework proposed by Guarnieri et al., has several limitations in practice.
%
For one the formulated model is based on the relational algebra. 
%
The reality, however is that DMBS are based on SQL, which differs from the relational algebra, and therefore posing several challenges for the implementation in a state-of-the-art DBMS.
%
Furthermore, due to the strong security requirements of the algorithm, a straight forward implementation would introduce an overhead of up to 200\%, thus the implementation needs to make use of heuristics to speed up the access control decision.
%

\remark{What are other limitations here? Until now I have only met these :)}

\remark{Should I include other proposed solutions here? If yes which ones or where do I find those?}

\smallskip
\noindent
{\bf Contributions.}
The original implementation of PostgreSQL access control is not handled uniformly, depending on the type of query there are different functions deciding if a query is authorized or not. 
%
Therefore the implementation of a new algorithm requires much more understanding of the backend architecture of PostgreSQL than actually need to implement and verify the algorithm.
%
The first contribution of this thesis is, that the codebase of PostgreSQL was refactored, moving all access control relevant code behind a uniform interface.
%
This provides the possibility of improving and extending access control without interfering with the internals of the DBMS.

As mentioned above the framework proposed by Guarnieri et al. lacks support for SQL. 
%
As a second contribution, the rules building this framework were extended to the semantics of SQL, therefore allowing it to be implemented in a DBMS.

Furthermore the algorithm has severe performance issues, another contribution of this thesis is to identify heuristics that can be used to increase the performance of this algorithm.
%
\remark{Here, you should list all the contributions of the thesis.
%
For each contribution, you should explain (1) why is it important and meaningful, and (2) how it improves the state-of-the-art.
%
 Note that contributions $\neq$ list of things that you did. A contribution is something new that improves the state-of-the-art. After this section, the reader must be convinced that the work in this thesis is an important contribution to humanity :-)}


\smallskip
\noindent
{\bf Organization.}
%
In the next chapter we will introduce some motivating examples of attacks from both families. 
%
In chapter 3 we will give background information about the advanced features that can be exploited. 
%
After that we will introduce the architecture of PostgreSQL and the new access control algorithm in chapter 4. 
%
In chapter 5 we will begin by showing the formal models of all aspects of the new security defintions and how they are extended from the relational algebra to SQL. 
%
Furthermore we will conduct experiments with state-of-the-art benchmark databases and queries in chapter 6. 
%
In chapter 7 we will introduce some related work to the thesis. 
%
Finally in chapter 8 the conclusions are drawn and future possibilities are discussed.