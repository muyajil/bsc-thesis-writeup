\section{Related Work}

The algorithms devised in this thesis are based on algorithms supporting the relational calclus by Guarnier et al. In~\cite{guarnieri2016strong} they showed the rewriting algorithm, inference rules and the authorization function for queries in the relational calculus including \texttt{SELECT}.
%
In~\cite{guarnieri2014optimal} the same authors show how queries can be executed in the presence of access control policies.

Further Rizvi et al.~\cite{rizvi2004extending} have identified the Non-truman and the Truman model of access control mechanisms.

In~\cite{wang2007correctness} the authors introduced a query modification algorithm that would also be suited for the Truman model.
%
However, their algorithm will, for each query $q$ issued by user $u$ replace each piece information obtained from $q$, which is not authorized to be read by $u$ by \texttt{NULL}, instead of just not returning it to the $u$.

Another query rewriting algorithm is found in~\cite{rizvi2004extending}. 
%
Here the authors use the concept of \emph{authorization views}.
%
For each relation they create an authorization view, which only contains the information that the database user is allowed to read.
%
In the Truman model they use these views by transparently substituting each referenced relation in the query with the corresponding authorization view.
%
Moreover they use these views in the Non-truman model to determine the authorization of queries, namely a query is authorized if it can be answered using only the authorization views.
%
Therefore the authorization of a query again boils down to query containment, i.e., a query $q$ is authorized if we can rewrite it to a query $q'$ that only uses the authorization views such that $q$ is equivalent to $q'$ in all database states.