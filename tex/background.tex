\section{Background}

In this Chapter we introduce the SQL access control model and outline how it is implemented in PostgreSQL.
%
The conclusions made in the following paragraphs are backed by an analysis of the backend code of PostgreSQL, since a proper documentation of the internals is missing.

\subsection{SQL access control model}

\remark{I'd first start by outlining the SQL access control model. There are privileges. Privileges are assigned/revoked using \texttt{GRANT} and \texttt{REVOKE} commands. This kind of things :-) This is essential to understand the rest (e.g., what are ACLs? how are they defined? what are the roles?)}

In the SQL access control model database users have privileges. 
%
Each privilege represents an op
%
Each privilege represents an operation that can be done in the database, such as, creating tables, creating triggers, update tables etc.
%
The privileges can be assigned using \texttt{GRANT} and revoked using \texttt{REVOKE}.
%
However the user issuing these queries, must have the privilege assigned to him.
%
This generally is the database administrator.
%
Privileges can 

\subsection{Implementation in PostgreSQL}
The administrator has to define an access control policy for the database. 
%
This policy defines how users can interact with the system.
%
A database object \remark{What is a db. object?} can refer to any structure within a database: tables, views, triggers, functions etc.
%
For each object the administrator has to define \emph{access control lists} (ACLs) for each database role.
%
The administrator can define an ACL for the public, a group\remark{,} or a user. 
%
These ACLs then define the privileges to a database object for a database role.
%
The privileges among others include creation of tables, views or triggers in the database or reading, altering or deleting of tuples from tables or views.
%
These privileges can be modified using \texttt{GRANT} or \texttt{REVOKE} commands, whereas the execution of those is also based on the ACLs \remark{the last part is a bit unclear}.
%
The possible privileges include insert, update, select, create, execute etc.\footnote{\url{http://www.postgresql.org/docs/9.5/static/sql-grant.html}}
%
Depending on the object different privileges are applicable. 
%
For example the execution privilege is only applicable to functions and triggers, whereas the insert privilege is only applicable to tables or views.
%
When issuing a query to the database, depending on the query, a function is executed which, among other things, retrieves the ACL for the objects referenced in the query and checks if the required privileges are available for the user issuing the command.