\section{Background}

Here, we introduce the SQL access control model and outline how it is implemented in PostgreSQL.
%
The conclusions made in the following paragraphs are backed by an analysis of the  PostgreSQL's code, since a proper documentation of the internals is missing.

\subsection{SQL access control model}

The SQL standard defines a number of privileges, where each privilege represents an operation that can be done in the database, such as creating tables, creating triggers, or updating tables.

Privileges can be assigned either on the database or on a \emph{database object}.
%
A database object refers to an internal structure of the database, such as a table, a trigger, or a view. 
%
When a user creates a database object, he is the owner of the object and therefore intrinsically receives all privileges on this object, including the privilege to delegate those to other users.
%
The privileges can be assigned to users using \texttt{GRANT} commands and revoked using \texttt{REVOKE} commands.
%
Note that the execution of  \texttt{GRANT} and \texttt{REVOKE} is regulated as well by the access control policy. 
%
A \texttt{GRANT} or \texttt{REVOKE} query on some database object is authorized if the user issuing the query is the database administrator, the owner of the database object, or if the user holds a \emph{grant option} on the privilege he is trying to assing or revoke.
%
A grant option can be assigned to a user together with a privilege.
%
The grant option gives the user the ability to delegate a privilege further.

In PostgreSQL, the SQL access control model is implemented as follows.
%
To simplify the creation of access control policies PostgreSQL supports the use of \emph{database roles}.
%
A database role can be either a single database user, a group of database users, or public. 
%
Groups of database users are used to manage privileges in bulk for users that enjoy similar privileges.
%
If a user was not assigned any specific privileges on a database objects either through group membership or directly, he automatically has all the privileges that are assigned to public.
%
Privileges are encoded using \emph{access control lists} (ACLs).
%
ACLs consist of several bitmasks.
%
Each bitmask represents the privileges assinged to a database role, further each bit within the mask represents one privilege.
%

\subsection{Implementation in PostgreSQL}

The implementation in PostgreSQL follows the SQL standard to large extents. 
%
Even though the implementation in PostgreSQL is rather close to the SQL standard, modifications are not straight forward, since, as we will see later, the different parts are spread in different modules of the DBMS.

When users submit queries to the database system, depending on the type of the query, an internal function is called, which among other things, retrieves the ACL for objects that are referenced in the query and checks the privileges.
%
Then the function will, according to the retrieved ACL either authorize the query or not.

\subsection{Algorithm proposed by Guarnieri et al.}

As mentioned in Chapter 1 Guarnieri et al. have devised a new access control mechanism.
%
One key component of this mechanism is determining if boolean queries in the relational calculus are secure with respect to the access control policy.
%
In the following we summarize this algorithm from~\cite{guarnieri2016strong}.
%
Since this undecidable for the relational calculus~\cite{guarnieri2014optimal}, they devised a function \emph{secure} which implements a sound and computable under-approximation of the security of boolean queries.
%
To compute this under-approximation they use query-rewriting.
%
A boolean query $\phi$ is rewritten to another query $\phi_{rw}$ such that if $\phi$ is not secure for the user $u$ issuing it to the database $db$ then $\phi_{rw}$ evaluates to $\top$ if issued to $db$ by $u$.
%
The query $\phi_{rw}$ is defined as $\lnot \phi^\top_u \land \phi^\bot_u$, where $\phi^\top_u$ and $\phi^\bot_u$ are defined as follows.
%
For each table or view $o$ in the database, the create additional views that represent every possible projection of $o$, together all these relations make up the set $O$.
%
Then they use query containment to compute, again for each table or view $o$, the sets $o^\top$, which contains all relations that contain $o$ and $o^\bot$, which similarly contains all relations that are contained in $o$.
%
Since query containment is undecidable~\cite{abiteboul1995foundations}, Guarnieri et al. have devised inference rules that produce a sound under-approximation of query containment.
%
Then $\phi^\top_u$ is computed by replacing each referenced relation $o$ with the conjunction of all relations $\in o^\top \cap \operatorname{auth}(u)$, where $\operatorname{auth}(u)$ returns all tables and views $\in O$ that are readable to $u$ according to the access control policy.
%
$\phi^\bot$ is computed similarly using $o^\bot$ and the disjunction instead of conjunction.

In chapter 5 we will show how we extend the inference rules for query containment and the rewriting algorithm to support SQL.