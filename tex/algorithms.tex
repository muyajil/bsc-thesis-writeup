\section{Algorithms}

\subsection{Query Containment}

The subject of query containment refers to the problem where for two queries $Q_1$ and $Q_2$ we want to answer if $Q_1$ is contained in $Q_2$ in all indistinguishable database states.
%
\remark{introduce indistinguishability of database states}
%
\remark{maybe reference to book for NP hardness} 

\subsection{Query Containment Approximation}

Since the decision of query containment is \emph{NP-hard}, i.e., not decidable in polynomial time, Guarnieri et al. have developed a rule set that over- respectively underapproximates the containment.
%
The rules were developed for the relational algebra, which as mentioned before differs from SQL. 
%
Thus during the thesis these rules were extended to support SQL.
%
In the following we will present the rule set for SQL. 
%
Let $T_i$ denote tables, $A_i$ sets of attributes within tables, $Q_i$ complete \texttt{SELECT} queries, and greek letters stand for boolean expressions.
\newline
{\renewcommand{\arraystretch}{1.5}
\begin{center}
	\begin{longtable}{ p{5mm} p{40mm} p{5mm} p{25mm} p{25mm} }
	(1) \label{id} &
	\texttt{SELECT $A_1$} \newline
	\texttt{FROM $T_1$} \newline
	\texttt{WHERE $\phi$} &
	$\subseteq$ &
	\texttt{SELECT $A_1$} \newline
	\texttt{FROM $T_1$} \newline
	\texttt{WHERE $\phi$}&
	\\

	(2) \label{and} &
	\texttt{SELECT $A_1$} \newline
	\texttt{FROM $T_1$} \newline
	\texttt{WHERE $\phi \land \psi$} &
	$\subseteq$ &
	\texttt{SELECT $A_1$} \newline
	\texttt{FROM $T_1$} \newline
	\texttt{WHERE $\phi$}&
	\\
	
	(3) \label{or} &
	\texttt{SELECT $A_1$} \newline
	\texttt{FROM $T_1$} \newline
	\texttt{WHERE $\phi$} &
	$\subseteq$ &
	\texttt{SELECT $A_1$} \newline
	\texttt{FROM $T_1$} \newline
	\texttt{WHERE $\phi \lor \psi$}&
	\\
	
	(4) \label{proj} &
	\texttt{SELECT $A_1$} \newline
	\texttt{FROM $T_1$} \newline
	\texttt{WHERE $\phi$} &
	$\subseteq$ &
	\texttt{SELECT $A_2$} \newline
	\texttt{FROM $T_1$} \newline
	\texttt{WHERE $\phi$}&
	if $A_1 \subseteq A_2$
	\\
	
	(5) \label{proj2} &
	\texttt{SELECT $A_1$} \newline
	\texttt{FROM $T_1$} \newline
	\texttt{WHERE $\phi$} &
	$\subseteq$ &
	\texttt{SELECT $A_2$} \newline
	\texttt{FROM $T_2$} \newline
	\texttt{WHERE $\phi$}&
	if $A_1 \subseteq A_2 \land T_1 \subseteq T_2$
	\\
	
	(6) \label{intersect1} &
	$Q_1$ \texttt{INTERSECTION} $Q_2$ &
	$\subseteq$ &
	$Q_1$&
	\\
	
	(7) \label{intersect2} &
	$Q_1$ \texttt{INTERSECTION} $Q_2$ &
	$\subseteq$ &
	$Q_2$&
	\\
	
	(8) \label{union1} &
	$Q_1$&
	$\subseteq$ &
	$Q_1$ \texttt{UNION} $Q_2$ &
	\\
	
	(8) \label{union2} &
	$Q_2$&
	$\subseteq$ &
	$Q_1$ \texttt{UNION} $Q_2$ &
	\\
	
	(9) \label{except} &
	$Q_1$ \texttt{EXCEPT} $Q_2$ &
	$\subseteq$ &
	$Q_1$&
	\\
	\end{longtable}
\end{center}

During this thesis a subset of the rules above were implemented.

\subsection{Authorization of \texttt{SELECT} queries}

In the example previously introduced in chapter 3, we showed that the current access control mechanism implemented in PostgreSQL is more restrictive than the mechanism implemented for this thesis.
%
The new mechanism relies on the concept of query containment.
%
For each \texttt{SELECT} query $Q$ issued by some user $u$ we do the following:
%
First we compute the overapproximation of the query using the query containment rules.
%
For each table or view $T$ referenced in the \texttt{SELECT} query, we compute the set $T^\bot$ that contains all tables or views that contain $T$.
%
After that we rewrite the original query $Q$ by replacing each referenced table or view with the intersection of all tables or views $\in T^\bot$ that, according to the access control policy, are readable to the user $u$.
%
This query is then submitted to the database and obtain the set of resulting tuples $Q^\bot$.
%
For the underapproximation we will, for each referenced table or view $T$, compute the set $T^\top$ containing all tables or views that are contained in $T$.
%
Then we will rewrite the original query $Q$ by replacing each referenced table or view with the union of all the tables or views $\in T^\top$ that, according to the access control policy, are readable to the user $u$.
%
After that we obtain the set of resulting tuples $Q^\top$.
%
The original query $Q$ is authorized if $Q^\top = Q^\bot$.

\smallskip
\noindent
{\bf Example:}
Recall the database setup with the following objects:
%
\begin{itemize}
	\item A table \texttt{Employees} containing all employees.
	\item A view \texttt{Programmers} defined by:
		\begin{verbatim}
		SELECT * 
		FROM Employees
		WHERE Job = "Programmer"
		\end{verbatim}
	\item A view \texttt{Oldprogrammers} defined by:
		\begin{verbatim}
		SELECT *
		FROM Programmers
		WHERE Age > 40
		\end{verbatim}
\end{itemize}
%
Further we have a user $u$ which has read access to the tables \texttt{Employees} and \texttt{Oldprogrammers}.
%
$u$ issues the following query $Q := $ \texttt{SELECT * FROM Programmers}, which by the current mechanism will not be authorized.
%
According to the rules above, specifically (2), we can deduce that \texttt{Programmers} $\subseteq$ \texttt{Employees} and \texttt{Oldprogrammers} $\subseteq$ \texttt{Programmers}.
%
\remark{I think we must include the table the user selects from, even if it is not authorized by the ACL, otherwise the example won't work...}