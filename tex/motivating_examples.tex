\section{Motivating Examples}

We show here how users can exploit advanced features of DBMSs to attack database integrity and confidentiality. 
%
The attacks are inspired by those presented in~\cite{guarnieri2016strong}.
%
These attacks are classified in two families: \emph{Integrity attacks} and \emph{Confidentiality attacks}.
%
In the former, the attacker is able to make unauthorized changes to the database. %, thus violating the integrity of the database from a security point of view.
%
In the latter, the attacker manages to learn sensitive information from the database, that according to the security policy, should not be disclosed to the attacker.
%
Guarnieri et al. have mounted these attacks manually against several widely used DBMS with shocking results: No existing access control mechanism is able to fully prevent these attacks.

\remark{Is it worth introducing integrity attacks since in the thesis we deal just with confidentiality?}

\subsection{Integrity Attacks}

Here we present integrity attacks. 
%
The following attacks use \texttt{INSERT}, \texttt{DELETE}, \texttt{GRANT}, and \texttt{REVOKE} commands together with views and triggers, which are procedures automatically executed by the DBMS in response to user commands.

In the first attack the attacker will use triggers to execute an unauthorized command on the database system. The trigger will be executed by an unaware user with a higher security clearance.

\begin{attack}
{\bf Trigger with activator's privileges:}
Consider a database with two tables $P$ and $S$ and two users $u$ and $w$. The attacker is the user $u$ whose goal is to delete the contents of $S$. The policy is as follows: $u$ is not authorized to alter $S$, $u$ can create triggers on $P$, and $w$ can read and modify $S$ and $P$. The attack is as follows:
\begin{enumerate}
\item $u$ creates the following trigger:
	\begin{verbatim}
	CREATE TRIGGER t ON P AFTER INSERT
    DELETE * FROM S
	\end{verbatim}
\item $u$ waits until $w$ insers a tuple into the table $P$. The trigger will then be invoked using $w$'s and $S$'s content will be deleted.
\end{enumerate}
\end{attack}
\remark{Explain triggers with activator's privileges.}

By replacing the \texttt{DELETE} command inside the trigger with a \texttt{GRANT} command an attacker can also escalate his privileges.

In the second attack the attacker will use views to escalate his privileges. He will delegate read privileges, without having the necessary permissions on the origin table.

\begin{attack}
{\bf Grating views:}
Consider a database with a table $S$ and two users $u$ and $v$, and the following policy: $u$ can create views and read $S$, but cannot delegate read permissions on $S$. $v$ cannot read $S$. The attacker is $u$ who will allow the unprivileged user $v$ to read the contents of $S$:
\begin{enumerate}
\item $u$ creates the view $V$:
	\begin{verbatim}
	CREATE VIEW V AS
	SELECT * FROM S;
	\end{verbatim}
\item $u$ issues the command:
	\begin{verbatim}
	GRANT SELECT ON V TO v;
	\end{verbatim}
\end{enumerate}
Now $v$ can read the contents of $S$ through $V$.
\end{attack}

\subsection{Confidentiality Attacks}

In the following we will present confidentiality attacks.



