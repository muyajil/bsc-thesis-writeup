\section{Motivating Example}

Here, we illustrate, using an example, the limitations of the algorithm proposed by the SQL standard for determining whether \texttt{SELECT} queries are secure.
%
We also show how our algorithm will address these limitations.


\smallskip
\noindent
{\bf Example:}
%
Consider a database with three tables \texttt{Programmers}, \texttt{Clerks}, and \texttt{HR}, containing the information of employees of the respective departements.
%
Assume, furthermore, that the database contains the following views:
\begin{itemize}
	\item \texttt{OldProgrammers} defined as \\ \texttt{SELECT * FROM Programmers WHERE Age > 40}
	\item \texttt{YoungProgrammers} defined as \\ \texttt{SELECT * FROM Programmers WHERE Age <= 40}
	\item \texttt{ProgrammerOrClerk} defined as \\ \texttt{(SELECT * FROM Programmers) UNION (SELECT * FROM Clerks)}
	\item \texttt{ProgrammerOrHR} defined as \\ \texttt{(SELECT * FROM Programmers) UNION (SELECT * FROM HR)}
\end{itemize}
%
Assume that there is a database user $u$ which has read access to all views, but does not have read access to the tables.
%
The user $u$ wants to find out who is employed as a Programmer and issues the following query $Q := $ \texttt{SELECT * FROM Programmers}. 
%
This query will not be authorized by the current access control mechanism implemented in PostgreSQL.
%
Since however $u$ has access to the views \texttt{OldProgrammers} and \texttt{YoungProgrammers} he can issue the query $Q_1 := $ \texttt{(SELECT * FROM OldProgrammers) UNION (SELECT * FROM YoungProgrammers)}, which in turn will be authorized by the access control mechanism in PostgreSQL.
%
Also the query $Q_2 := $ \texttt{(SELECT * FROM ProgrammerOrClerk) INTERSECTION (SELECT * FROM ProgrammerOrHR)} will be authorized by PostgreSQL.
%
It is obvious that all three queries return the same results.
%
This behaviour of authorizing \texttt{SELECT} queries is a key part in the algorithm proposed in~\cite{guarnieri2016strong}.
%
In Chapter 5 we will come back to this example and show how we authorize this query with the new access control mechanism implemented in PostgreSQL.
