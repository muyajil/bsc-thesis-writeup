\section{Motivating Examples}

We show here how users can exploit advanced features of DBMSs to attack database confidentiality
%
The following attacks were first presented in~\cite{guarnieri2016strong}
%
In these attacks, the attacker manages to learn sensitive information from the database, that according to the security policy, should not be disclosed to the attacker.
%
Guarnieri et al. have mounted these attacks manually against several widely used DBMS with shocking results: No existing access control mechanism is able to fully prevent these attacks.

\subsection{Confidentiality Attacks}

In the following we will present confidentiality attacks. 
%
We will use \texttt{SELECT} and \texttt{INSERT} commands in combination with triggers and integrity constraints to infer classified information.
%
According to~\cite{guarnieri2016strong} integrity constraints are invariants that must be satisfied for a database state to be considered \emph{valid}.
%
\emph{Integrity constraint violations} arise when the execution of an SQL command leads the database from a valid state into an invalid one.
%
\begin{attack}
{\bf Table updates and integrity constraints.~\cite{guarnieri2016strong}}
%
Consider a database with two tables $P$ and $S$. 
%
Suppose the primary key for both tables is the user's identifier.
%
Furthermore the set of user identifiers in $S$ is contained in the set of user identifiers in $P$, i.e., there is a foreign key from $S$ to $P$.
%
The attacker is user $u$, whose goal is to learn whether \texttt{Bob} is in $S$.
%
The access control policy is as follows: $u$ can read from $P$ and insert tuples into $S$.

The attacker can learn whether \texttt{Bob} is in $S$ as follows:
\begin{enumerate}
	\item He reads $P$ and learns \texttt{Bob}'s identifier.
	\item He issues an \texttt{INSERT} statement into $S$ using \texttt{Bob}'s id.
	\item If \texttt{Bob} is already in $S$, then $u$ gets an error message about the primary key's violation. Alternatively there is no violation and $u$ learns that \texttt{Bob} is not in $S$.
\end{enumerate}
\end{attack}
%
The first step of this attack includes a \texttt{SELECT} statement to the database, from which the attacker knows that it does not change the database state.
%
The second step is actually the key of this attack, from the information the attacker has learned from the first step, he can issue the second step at any point in time.
%
Everytime the above \texttt{INSERT} command is issued, the attacker learns whether \texttt{Bob} is in the table $S$ in the database state that directly precedes the transaction containing the \texttt{INSERT} command.
%
In the next attack the attacker learns sensitive information by exploiting triggers.
%
The trigger is executed with owner's privileges, meaning it is executed with the same privileges as the user, which has created the trigger.

\begin{attack}
{\bf Triggers with owner's privileges.~\cite{guarnieri2016strong}}
%
Consider a database with tables $P$, $T$, and $N$.
%
The attacker is user $u$, who wishes to learn whether $v$ is in $T$.
%
The policy is as follows:
%
$u$ is not authorized to read table $T$, and he can read and modify tables $P$ and $N$.
%
Moreover, the following trigger has been defined by the administrator:
%
\begin{verbatim}
CREATE TRIGGER t on P AFTER INSERT FOR EACH ROW
	IF EXISTS(SELECT * FROM T where id = new.id)
		INSERT new.id INTO N;
\end{verbatim}
%
The attack is as follows:
%
\begin{enumerate}
	\item $u$ deletes $v$ from $N$.
	\item $u$ issues the command: \texttt{INSERT INTO P VALUES(v)}
	\item $u$ checks the table $N$. If it contains $v$'s id, then $v$ is in $T$. Otherwise $v$ is not in $T$.
\end{enumerate}
%
\end{attack}
%
In the first step the attacker issues a \texttt{DELETE} statement, from which he learns that $v$ cannot be in the table $N$ in the database state directly subsequent to the \texttt{DELETE} statement.
%
The attacker knows about the trigger in the database, and therefore can reason, that the trigger is always executed after an \texttt{INSERT} statement on table $P$.
%
He further knows, that the trigger is executed in the same transaction as the \texttt{INSERT} statement, therefore after issuing the \texttt{INSERT} into table $P$ it is not possible that another \texttt{INSERT} into $N$ is executed first.
%
Therefore the attacker knows, that if $v$ is not in $N$ in the database state where he issues the \texttt{INSERT} statement, $v$ is inserted into $N$ if and only if $v$ is already in $T$.

\remark{What if between the delete and insert there is a command \texttt{INSERT INTO N VALUES(v)}?}

