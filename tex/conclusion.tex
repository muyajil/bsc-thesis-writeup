\section{Conclusion}

As shown in~\cite{guarnieri2016strong}, current database access control mechanisms are not secure.
%
Therefore, databases need new, stronger access control mechanisms to prevent attacks threatening both database integrity and confidentiality. %is here and database vendors need to address this fact in the future.
%
In this thesis, we address two main issues that 
%
First, we refactor the code of a modern open-source DBMS, namely PostgreSQL, in such a way to ease the implementation of new access control mechanisms.
%
The current PostgreSQL's architecture is not adequate to support these new mechanisms, since access control is closely tied with the database internals.
%
We performed  major modifications to the PostgreSQL's code addressing access control.
%
\remark{A few more information here?}
%
These modifications pave the way for the implementation of new access control mechanisms by providing developers with a uniform interface and clear code segregation.


Second, we developed an algorithm for deciding whether queries are secure or not.
%
Our algorithm directly applies to the SQL language.
%
As shown also in~\cite{guarnieri2016strong}, this kind of algorithms is an important building block for novel access control mechanisms that aim at prevent the leakage of sensitive information.
%
Existing algorithms, such as those proposed by the SQL standard and implemented in PostgreSQL, are overly restrictive.
%
Using them as building blocks for novel mechanisms would result in almost unusable systems.
%
Our solution, in contrast, exploits advanced query rewriting techniques to determine whether queries are secure or not. 
%
As a result, it is more permissive than existing solutions.

\remark{Conclusion should also discuss future work/possible extension/ lesson learned. How do you think other people will use what you've done.}


\remark{Most of this discussion should probably go in the related work section.}


\remark{on what do we want to focus further here?}