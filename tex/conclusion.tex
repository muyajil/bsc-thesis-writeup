\section{Conclusion}

As shown in~\cite{guarnieri2016strong}, current database access control mechanisms are not secure.
%
Therefore, databases need new, stronger access control mechanisms to prevent attacks threatening both database integrity and confidentiality.
%
In this thesis, we address two main issues.
%
First, we refactor the code of a modern open-source DBMS, namely PostgreSQL, in such a way to ease the implementation of new access control mechanisms.
%
The current PostgreSQL's architecture is not adequate to support these new mechanisms, since access control is closely tied with the database internals.
%
We performed  major modifications to the PostgreSQL's code addressing access control.
%
We changed the behaviour of certain Backend modules such as the executor, further we moved all access control relevant code to a stand-alone module. 
%
Moreover we created an interface to said module.
%
These modifications pave the way for the implementation of new access control mechanisms by providing developers with a uniform interface and clear code segregation.


Second, we developed an algorithm for deciding whether queries are secure or not.
%
Our algorithm directly applies to the SQL language.
%
As shown also in~\cite{guarnieri2016strong}, this kind of algorithms is an important building block for novel access control mechanisms that aim at prevent the leakage of sensitive information.
%
Existing algorithms, such as those proposed by the SQL standard and implemented in PostgreSQL, are overly restrictive.
%
Using them as building blocks for novel mechanisms would result in almost unusable systems.
%
Our solution, in contrast, exploits advanced query rewriting techniques to determine whether queries are secure or not. 
%
As a result, it is more permissive than existing solutions.

As shown in Chapter 5 our algorithm is easy extensible to support the Truman model.
%
However, the implementation of the Truman model would require further changes to the executor and transistions between modules.
%
Future work can include the improvement of the rewriting algorithm, such that it generates better approximations, by implementing all inference rules from Chapter 5.

This thesis provides a first step on a long way to implement stronger database access control mechanisms.
%
Further it also allows to tackle this way step by step by providing a mechanism that can gradually be changed to support more query types, such as, \texttt{INSERT}, \texttt{UPDATE}, or \texttt{GRANT}.
