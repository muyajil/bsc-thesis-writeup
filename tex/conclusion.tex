\section{Conclusion}

As shown in~\cite{guarnieri2016strong} current database access control mechanisms are not secure.
%
Therefore the need for new, stronger access control mechanisms is here and database vendors need to address this fact in the future.
%
This thesis however, shows that the implementation of such mechanisms might interfere with the database internals, therefore major modifications are needed in existing DBMS such as PostgreSQL on how access control decisions are made.
%
The modifications made to the PostgreSQL internals during this thesis however prepare the implementation of new access control mechanisms by providing developers with a uniform interface and clear code segregation.
%
The algorithms devised in this thesis are constructed under the \emph{Truman model}.
%
It specifies an access control mechanism as a component which decides if a query is optimized or not, and therefore returns $\bot$ or $\top$.
%
The current implementations of these mechanisms either interrupt the execution of the query and notify the user if a query was not authorized or do not interrupt the execution of the query if the query was authorized.
%
The Truman model is the most used model in database access control mechanisms.
%
The \emph{Non-truman model} however uses access control mechanisms with another approach, namely the access control internally executes the query, and only returns the tuples, which are readable for the user according to the access control policy.
%
The user issuing a query will not notice if the query he issued is not authorized for him.
%
Our algorithm is extensible, further it would even be better suited to the Non-truman model.
%
We under-approximate the tuples resulting from a query during the algorithm that returns the access control decision, and we further know that the under-approximation is secure with respect to the access control policy.
%
Therefore it is secure to return the under-approximation to the user.
%
However our implementation contains a subset of the introduced inference rules for containment, implementing all of them would result in a better under-approximation.
\remark{on what do we want to focus further here?}